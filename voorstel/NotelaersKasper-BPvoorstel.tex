%==============================================================================
% Sjabloon onderzoeksvoorstel bachproef
%==============================================================================
% Gebaseerd op document class `hogent-article'
% zie <https://github.com/HoGentTIN/latex-hogent-article>

% Voor een voorstel in het Engels: voeg de documentclass-optie [english] toe.
% Let op: kan enkel na toestemming van de bachelorproefcoördinator!
\documentclass{hogent-article}

% Invoegen bibliografiebestand
\addbibresource{voorstel.bib}

% Informatie over de opleiding, het vak en soort opdracht
\studyprogramme{Professionele bachelor toegepaste informatica}
\course{Bachelorproef}
\assignmenttype{Onderzoeksvoorstel}
% Voor een voorstel in het Engels, haal de volgende 3 regels uit commentaar
% \studyprogramme{Bachelor of applied information technology}
% \course{Bachelor thesis}
% \assignmenttype{Research proposal}

\academicyear{2025-2026} % TODO: pas het academiejaar aan

% TODO: Werktitel
\title{Hoe kan een digitale tool studenten met ASS en een gemiddeld tot hoog IQ ondersteunen bij het verbeteren van zelf-monitoring binnen zelfregulerend leren in het hoger onderwijs?}

% TODO: Studentnaam en emailadres invullen
\author{Kasper Notelaers}
\email{kasper.notelaers@student.hogent.be}

% TODO: Medestudent
% Gaat het om een bachelorproef in samenwerking met een student in een andere
% opleiding? Geef dan de naam en emailadres hier
% \author{Yasmine Alaoui (naam opleiding)}
% \email{yasmine.alaoui@student.hogent.be}

% TODO: Geef de co-promotor op
\supervisor[Co-promotor]{W. De Bent (wim.debent@hetraster.be)}

% Binnen welke specialisatierichting uit 3TI situeert dit onderzoek zich?
% Kies uit deze lijst:
%
% - Mobile \& Enterprise development
% - AI \& Data Engineering
% - Functional \& Business Analysis
% - System \& Network Administrator
% - Mainframe Expert
% - Als het onderzoek niet past binnen een van deze domeinen specifieer je deze
%   zelf
%
\specialisation{Mobile \& Enterprise development}
\keywords{zelfregulerend leren, zelf-monitoring, gamification, autisme spectrum stoornis, digitale tools, executieve functies}

\begin{document}

\begin{abstract}
Studenten met autismespectrumstoornis (ASS) ervaren vaak tekorten bij executieve functies. Dit zijn cognitieve vaardigheden die toelaten om doelgericht te handelen en zich sociaal aangepast te gedragen. De drie kern-executieve functies werkgeheugen, impulscontrole en cognitieve flexibiliteit vormen de basis voor hogere orde executieve functies zoals plannen en organiseren, probleemoplossend denken en gedragsevalatie. Deze  zijn cruciaal bij zelfregulerend leren en kunnen bij een deficit het functioneren van studenten met ASS belemmeren. 
Deze studie onderzoekt de impact van executieve functieproblemen op studiegedrag en academische prestaties in het hoger onderwijs. Daarnaast brengt het analoge en digitale hulpmiddelen die gericht zijn op zelfmonitoring en zelfregulatie in kaart. 
Analoge tools zoals papieren studieplanners, leerdagboeken en checklists maken doelen en voortgang expliciet en ondersteunen daarmee het structureren van het leerproces. Digitale hulpmiddelen variëren van reminder-apps (zoals I-Connect) en gamified takenlijsten (zoals Habitica) tot educatieve zelfregulerend leren-apps (zoals Ace Your Self-Study) die studenten stapgsgewijs begeleiden bij het plannen, monitoren en evalueren van hun leeractiviteiten. 
Op basis van de requirementsanalyse wordt een Proof of Concept ontwikkeld. Dit prototype heeft een exploratief karakter en dient om te verkennen hoe dit de geïdentificeerde ondersteuningsbehoeften van studenten met ASS kan ondersteunen, zonder hierbij direct in te zetten op het meten van effecten op gedrag of studieresultaten.
\end{abstract}


\tableofcontents

% De hoofdtekst van het voorstel zit in een apart bestand, zodat het makkelijk
% kan opgenomen worden in de bijlagen van de bachelorproef zelf.
%---------- Inleiding ---------------------------------------------------------

% TODO: Is dit voorstel gebaseerd op een paper van Research Methods die je
% vorig jaar hebt ingediend? Heb je daarbij eventueel samengewerkt met een
% andere student?
% Zo ja, haal dan de tekst hieronder uit commentaar en pas aan.

%\paragraph{Opmerking}

% Dit voorstel is gebaseerd op het onderzoeksvoorstel dat werd geschreven in het
% kader van het vak Research Methods dat ik (vorig/dit) academiejaar heb
% uitgewerkt (met medesturent VOORNAAM NAAM als mede-auteur).
% 

\section{Inleiding}%
\label{sec:inleiding}
Jongvolwassenen met een autismespectrumstoornis (ASS) vertonen vaak hardnekkige problemen met executieve functies, waaronder planning, organisatie, timemanagement, taakinitiatie en impulscontrole. 
Empirisch onderzoek wijst er op dat zij op deze domeinen significant zwakker presteren dan leeftijdsgenoten zonder ASS. Deze beperkingen bemoeilijken het systematisch structureren van studietaken en het doelgericht aansturen van leeractiviteiten, wat zich in het hoger onderwijs uit in gemiste deadlines, onvoldoende geordend studiemateriaal, vertraagde studievoortgang en minder goede academische prestaties.

Executieve functies vormen de kern van zelfregulerend leren, een cyclisch proces van plannen, monitoren en bijsturen van het eigen leerproces. 
Voor studenten met ASS blijkt dit proces bijzonder complex en uitdagend, mede door autismespecifieke kenmerken zoals cognitieve rigiditeit, uitgesproklen detailgerichtheid en een deficit in metacognitieve vaardigheden. Deze factoren bemoeilijken het accuraat inschatten van eigen leerproces en het effectief inzetten van zelfmonitoringstrategieën waardoor bijsturing van het leerproces vaak uitblijft. 

Tegen deze achtergrond groeit de aandacht voor ondersteunende hulpmiddelen die gericht zijn op het versterken van zelfmonitoring en zelfregulatie, waaronder digitale toepassingen. 
Dergelijke hulpmiddelen kunnen fungeren als externe ondersteuning van executieve functies door structuur aan te brengen, voortgang expliciet te maken en studenten met ASS te ondersteunen in het bewust plannen en reguleren van hun leerproces.

Dit onderzoek richt zich daarom op de volgende deelvragen:
\begin{itemize}
\item Welke problemen ervaren jongvolwassenen met ASS in executieve functies?

\item Hoe beïnvloeden executieve-functieproblemen het academisch functioneren van ASS-studenten in het hoger onderwijs?

\item Welke uitdagingen ervaren jongvolwassenen met ASS bij het toepassen van zelfregulerend leren?

\item Welke ASS-kenmerken bemoeilijken zelfmonitoring tijdens het studeren?

\item Welke niet-digitale hulpmiddelen ondersteunen zelfmonitoring en zelfregulatie?

\item Welke digitale hulpmiddelen bestaan er ter ondersteuning van zelfmonitoring?

\item Hoe ondersteunen digitale tools zoals I-Connect en Habitica het zelfmonitoringsproces?

\item Op welke manier ondersteunen niet-digitale hulpmiddelen het zelfmonitoringsproces?

\item Aan welke functionele en niet functionele vereisten moet een hulpmiddel voldoen?
\end{itemize}

%Waarover zal je bachelorproef gaan? Introduceer het thema en zorg dat volgende zaken zeker duidelijk aanwezig zijn:

%\begin{itemize}
%  \item kaderen thema
%  \item de doelgroep
%  \item de probleemstelling en (centrale) onderzoeksvraag
%  \item de onderzoeksdoelstelling
%\end{itemize}

%Denk er aan: een typische bachelorproef is \textit{toegepast onderzoek}, wat betekent dat je start vanuit een concrete probleemsituatie in bedrijfscontext, een \textbf{casus}. Het is belangrijk om je onderwerp goed af te bakenen: je gaat voor die \textit{ene specifieke probleemsituatie} op zoek naar een goede oplossing, op basis van de huidige kennis in het vakgebied.

%De doelgroep moet ook concreet en duidelijk zijn, dus geen algemene of vaag gedefinieerde groepen zoals \emph{bedrijven}, \emph{developers}, \emph{Vlamingen}, enz. Je richt je in elk geval op it-professionals, een bachelorproef is geen populariserende tekst. Eén specifiek bedrijf (die te maken hebben met een concrete probleemsituatie) is dus beter dan \emph{bedrijven} in het algemeen.

%Formuleer duidelijk de onderzoeksvraag! De begeleiders lezen nog steeds te veel voorstellen waarin we geen onderzoeksvraag terugvinden.

%Schrijf ook iets over de doelstelling. Wat zie je als het concrete eindresultaat van je onderzoek, naast de uitgeschreven scriptie? Is het een proof-of-concept, een rapport met aanbevelingen, \ldots Met welk eindresultaat kan je je bachelorproef als een succes beschouwen?

%---------- Stand van zaken ---------------------------------------------------

\section{Literatuurstudie}%
\label{sec:literatuurstudie}

In deze bachelorproef wordt zelfmonitoring gedefinieerd als het systematisch observeren, registreren en evalueren van eigen leer- en taakgedrag tijdens het leerproces. \autocite{WhittenburgHamMcDonough2021}. Zelfmonitoring vormt een kerncomponent van zelfregulerend leren.

\subsection{Executieve functies bij jongvolwassenen met ASS}

Jongvolwassenen met autisme vertonen frequent tekorten in diverse executieve functies, zoals planning, organisatie, cognitieve flexibiliteit, werkgeheugen en impulscontrole. Deze beperkingen beïnvloeden in het bijzonder hun vermogen tot zelfmonitoring, dat centraal staat in dit onderzoek \autocite{VanEsch2022, meerman2020inclusief, duncan2022school}. 
Problemen met planning, organisatie en flexibiliteit maken het voor studenten met ASS lastig om studietaken te structureren en overzicht te behouden, terwijl beperkte cognitieve flexibiliteit het herkennen van afwijkingen in het leerproces en het bijsturen van strategieën belemmert. \autocite{meerman2020inclusief}. 
Tekorten in werkgeheugen, taakinitiatie en impulscontrole versterken deze problematiek doordat studenten moeite hebben om bij te houden welke stappen reeds zijn uitgevoerd, om tijdig aan taken te beginnen en om afleidend of  impulsief gedrag te reguleren \autocite{duncan2022school}. 
Zelfrapportages tonen bovendien aan dat jongvolwassenen met hoogfunctionerend autisme in het dagelijks leven aanzienlijk meer problemen ervaren met executieve functies dan hun neurotypische leeftijdsgenoten \autocite{dijkhuis2017selfregulation}.
Deze bevindingen wijzen erop dat beperkingen in plannen, organiseren, timemanagement, cognitieve flexibiliteit en zelfmonitoring kenmerkend zijn bij studenten met ASS.


\subsection{Executieve functies en academisch functioneren in het hoger onderwijs}

In het hoger onderwijs zijn goed ontwikkelde executieve vaardigheden cruciaal voor academisch succes. Studenten moeten zelfstandig studeren, studietaken inplannen, deadlines halen en initiatied nemen bij individuele en groepsopdrachten. \autocite{dijkhuis2017selfregulation}
Wanneer deze executieve functies beperkt zijn, ontstaan concrete studieproblemen. Gebrekkige planningsvaardigheden kunnen leiden tot gemiste colleges of opdrachten, terwijl problemen met timemanagement resulteren in problemen bij examens en het niet tijdig afronden van projecten. \autocite{meerman2020inclusief}
Een review toont aan dat studenten met ASS vooral moeite ervaren  met het organiseren van studiemateriaal, het prioriteren van opdrachten, effectief studeren en het opdelen van complexe taken. Dit zijn precies de vaardigheden die essentieel zijn voor succesvol functioneren in het hoger onderwijs. \autocite{duncan2022school}
Als gevolg hiervan lopen studenten met ASS gemiddeld 2 tot 3 jaar achter op hun neurotypische klasgenoten in studievoortgang. 
Problemen in executief functioneren vertalen zich dus rechtstreeks in inefficiënte studiegewoonten en verminderde academische prestaties, terwijk sterk ontwikkelde executieve functies juist een beschermende factor vormen bij deze groep. \autocite{duncan2022school}


\subsection{Zelfregulerend leren bij studenten met ASS}

Zelfregulerend leren vereist dat studenten hun eigen leerproces doelgericht plannen, continu monitoren en waar nodig bijsturen. Beperkingen in metacognitieve vaardigheden beïnvloeden dit proces rechtstreeks, aangezien studenten met ASS vaak moeite hebben om hun eigen begrip en voortgang accuraat te observeren, te registreren en te evalueren, wat het bijsturen van leerstrategieën bemoeilijkt \autocite{VanEsch2022}.
In combinatie met verminderde planningscapaciteit en cognitieve rigiditeit resulteert dit in weinig effectief zelfgestuurd leren. Studenten stellen minder vaak concrete doelen, slaan essentiële studiestappen over en signaleren pas laat dat ze achterop raken. Dökkerbächer en Bregulla (2024) benadrukken dat executieve functies zoals metacognitie, planning en impulscontrole de drijvende kracht vormen van zelfregulatie. \autocite{duncan2022school}
Aangezien deze functies bij ASS vaak minder sterk ontwikkeld zijn, wordt het inzetten van strategieën zoals tijdsplanning, voortgangscontrole en reflectie bemoeilijkt. De combinatie van zwakke planningsvaardigheden, beperkte cognitieve flexibiliteit en verminderd zelfbewustzijn maakt het voor jongeren met ASS bijzonder moeilijk om het leerproces zelfstandig en doelgericht te structureren. \autocite{DörrenbächerUlrichBregulla2024}
Deze beperkingen benadrukken het belang van ondersteunende structuren die zelfmonitoring expliciet faciliteren, aangezien spontane zelfregulatie voor deze doelgroep minder vanzelfsprekend is.

\subsection{Autistische kenmerken als belemmering voor zelfmonitoring}

Specifieke kenmerken van autisme, waaronder beperkte cognitieve flexibiliteit en een voorkeur voor vaste routines, vormen een bijkomende belemmeringen voor effectief zelfmonitorend gedrag. Studenten volgen vaak één strategie en merken onvoldoende op wanneer deze niet langer effectief is, waardoor bijsturing van het leerproces uitblijft \autocite{meerman2020inclusief}. Zo vergeten ze bijvoorbeeld om pauzes in te lassen of om hun studietijd bij te houden, waardoor planning faalt.  \autocite{meerman2020inclusief}
Daarnaast dragen een zwakke centrale coherentie (focus op detail in plaats van het geheel) en verminderd inlevingsvermogen ervoor dat studenten niet goed inschatten hoe het studiegedrag verloopt. Metacognitieve beperkingen betekenen het niet overzien wanneer iets niet begrepen is, of achterlopen op schema. \autocite{kool2025metacognitie} 
Onderzoek bij studenten met hoogfunctionerend autisme wijst erop dat problemen met gedragsmonitoring en het inschatten van de gevolgen van eigen gedrag samenhangen met achterblijvende academische resultaten. \autocite{dijkhuis2017selfregulation}
Autisme-kenmerken zoals rigiditeit, beperkte zelfreflectie en detailgerichte informatieverwerking bemoeilijken het toepassen van zelfmonitoringstrategieën bij het studeren zoals het bewaken van de tijd, het controleren van werk en het herzien van leerdoelen. Hierdoor kunnen studenten met ASS vaak niet volledig profiteren van hun leeractiviteiten.
Zelfmonitoring vormt een essentieel mechanisme binnen zelfregulerend leren. 
Onderzoek van Isaacson en Fujita (2006) toont aan dat studenten die hun begrip, voortgang en prestaties accuraat monitoren, beter in staat zijn hun leerstrategieën bij te sturen en daardoor hogere academische resultaten behalen. 
Wanneer zelfmonitoring ontbreekt, verloopt dit bijsturingsproces inefficiënt, wat zich vertaalt in minder effectieve leerstrategieën en lagere studieprestaties \autocite{IsaacsonFujita2006}.
Voor studenten met ASS wordt dit probleem versterkt door beperkingen in executieve functies, die als randvoorwaarden fungeren voor effectief zelfmonitorend gedrag. 
Dit benadrukt voor deze doelgroep het belang en de noodzaak van ondersteunende structuren die zelfmonitoring binnen zelfregulerend leren kunnen versterken 

\subsection{Hulpmiddelen ter ondersteuning van zelfmonitoring}

Zowel niet-digitale als digitale hulpmiddelen kunnen worden ingezet ter ondersteuning van zelfmonitoring binnen zelfregulerend leren. 
Deze tools verschillen echter in aard, timing en mate van ondersteuning, wat relevant is voor studenten met ASS die moeite hebben met planning, monitoring en flexibel bijsturen van hun leerproces.

\subsubsection{Niet-digitale hulpmiddelen}

Niet-digitale hulpmiddelen zoals papieren planners, leerdagboeken en checklists worden traditioneel gebruikt om zelfmonitoring te ondersteunen. Papieren agenda’s en studieplanners helpen studenten om leerdoelen en studietaken expliciet te formuleren en hun planning overzichtelijk vast te leggen. 
In leerdagboeken noteren studenten welke activiteiten ze hebben uitgevoerd, welke strategieën zijn gebruikt en in welke mate doelen zijn behaald. Dit bevordert zelfreflectie en bewustwording van het eigen leerproces \autocite{Cheng2025}. 
Onderzoek toont aan dat (digitale en analoge) leerdagboeken het stellen van doelen en het bijhouden van voortgang ondersteunen, beide kerncomponenten van zelfregulerend leren \autocite{Cheng2025}.
Daarnaast worden zelfmonitoringschecklists en reflectiekaarten frequent gebruikt. Hierbij registreren studenten hun eigen taakgedrag op bepaalde momenten. Zelfmonitoring-interventies bestaan doorgaans uit het definiëren van doelgedrag, gevolgd door zelfbeoordeling en zelfregistratie \autocite{Davis2016}. 
Meta-analyses tonen aan dat deze zelfregistratie een essentieel onderdeel vormt van effectieve zelfmonitoringinterventies \autocite{Davis2016}. 
Dergelijke papieren hulpmiddelen zijn laagdrempelig en bevorderen reflectie achteraf. Ook visuele planners, taakstroken en fysieke timers kunnen structuur bieden. Door taken en tijdsblokken visueel weer te geven, ondersteunen zij planning en tijdsbewustzijn. Deze hulpmiddelen zijn vereisen echter een hoge mate van zelfinitiatie van de student en consistent gebruik. \autocite{HallezVallier2025}


\subsubsection{Digitale hulpmiddelen}

Digitale hulpmiddelen bouwen voort op dezelfde principes, maar bieden aanvullende functionaliteiten die beter aansluiten bij de noden van studenten met ASS. Apps zoals I-Connect maken gebruik van real-time visuele prompts die studenten op vaste momenten vragen of zij met hun taak bezig zijn. Door deze frequente check-ins en directe registratie blijkt het on-task-gedrag aanzienlijk toe te nemen \autocite{Li2023}. 
Dit is vooral relevant voor studenten met ASS die baat hebben bij concrete, herhaalde prikkels.
Gamified apps zoals Habitica, Forest en Focusverse koppelen taakuitvoering aan beloningen of visuele feedback. 
Hierdoor wordt motivatie gestimuleerd en wordt voortgang expliciet zichtbaar gemaakt \autocite{Lieder2024, ForestAppSiho2025, FocusverseProductHunt2025}. 
Hoewel gamificatie niet voor elke student even effectief is, kan het helpen om zelfmonitoring vol te houden bij routinetaken. Gamificatie kan zelfmonitoring ondersteunen door leerdoelen en voortgang visueel en expliciet te maken via elementen zoals voortgangsbalken, niveaus en directe feedback, waardoor studenten beter zicht krijgen op hun leerproces \autocite{Hamari2014, Dicheva2015}. 
Tegelijkertijd tonen studies aan dat veel gamified systemen sterk leunen op extrinsieke beloningen, zoals punten, badges en ranglijsten, wat de intrinsieke motivatie en zelfregulatie kan ondermijnen wanneer feedback als controlerend of competitief wordt ervaren \autocite{DeciKoestnerRyan1999, RyanDeci2017}. 
Voor studenten met een autismespectrumstoornis zijn deze risico’s extra relevant, aangezien sensorisch intensieve en competitieve leeromgevingen kunnen leiden tot overprikkeling, stress en een verhoogde afhankelijkheid van externe feedback \autocite{Grynszpan2014, Parsons2017}. 
Gamificatie kan zelfmonitoring daarentegen versterken wanneer zij voorspelbare, transparante en niet-competitieve feedback biedt en expliciet wordt ingezet ter ondersteuning van reflectie en voortgangsbewustzijn, in plaats van als primair motivatiemechanisme \autocite{Hamari2014, RyanDeci2017}.
Meer inhoudelijk gerichte apps zoals Ace Your Self-Study ondersteunen expliciet alle fasen van zelfregulerend leren: planning, uitvoering en reflectie.  Studenten stellen vooraf doelen en strategieën in, ontvangen tijdens het studeren reminders en reflecteren nadien op hun prestaties \autocite{Baars2022}. 
Door deze cyclische aanpak wordt zelfmonitoring actief geïntegreerd in het leerproces. Daarnaast bestaan er andere tools zoals Self-Monitor Habit Changer en TickTick, die frequente zelfmonitoring makkelijker maken via instelbare herinneringen, automatische registratie en statistische overzichten.  Deze tools verminderen de cognitieve belasting die gepaard gaat met handmatige registratie en maken patronen in studiegedrag zichtbaar \autocite{SelfMonitorHabitChangerAppStore, TickTickWebsite2026}. 
Ook platforms zoals Selbstlernen richten zich expliciet op het versterken van zelfregulatievaardigheden via digitale ondersteuning in een inclusieve onderwijscontext \autocite{SelbstlernenAppInklusionNetwork2025}.

\subsection{Samenvatting}

Samenvattend ondersteunen niet-digitale hulpmiddelen zoals planners, leerdagboeken en checklists vooral het expliciteren van doelen en taken en reflectie achteraf. Ze zijn eenvoudig en concreet, maar vereisen een hoge mate van zelfinitiatie en consistent gebruik. Digitale hulpmiddelen bieden daarentegen duidelijke meerwaarde door real-time ondersteuning, automatische registratie, en visualisatie van voortgang. 
Deze eigenschappen sluiten beter aan bij de executieve beperkingen van studenten met ASS, zoals moeite met planning, werkgeheugen en zelfmonitoring. Digitale tools zijn daarom een veelbelovende basis voor het ontwikkelen van een proof-of-concept die zelfmonitoring binnen zelfregulerend leren op een toegankelijke en ondersteunende manier versterkt.










%Zowel niet-digitale als digitale hulpmiddelen kunnen worden ingezet om zelfmonitoring binnen zelfregulerend leren te ondersteunen. 
%Deze tools verschillen echter in aard, timing en mate van ondersteuning, wat relevant is voor studenten met ASS die moeite hebben met planning, monitoring en flexibel bijsturen van hun leerproces.

%Niet-digitale hulpmiddelen

%Papieren planners en leerdagboeken: Traditionele hulpmiddelen zoals papieren agenda’s, studieplanners of leerdagboeken helpen studenten om doelen en studie-taken uit te schrijven. In een leerdagboek (self-learning diary) noteert een student bijvoorbeeld wekelijks wat hij heeft gestudeerd, welke strategieën zijn gebruikt en in hoeverre doelen zijn behaald. Zulke zelfreflectie-instrumenten bevorderen zelfmonitoring doordat ze studenten bewust maken van hun planning en voortgang. \autocite{Cheng2025}
%Onderzoek toont aan dat (digitale) leerdagboeken het stellen van leerdoelen en het bijhouden van de voortgang faciliteren – beide cruciale componenten van zelfregulatie. \autocite{Cheng2025}
%Ook zonder app kunnen leerlingen bijgehouden schemaboekjes of checklist-formulieren gebruiken om doelen te stellen en aantekeningen te maken over hoever ze zijn gekomen.

%Zelf-monitoringschecklists en reflectiekaarten: Een veelgebruikte analoge methode is het invullen van checklists of formulieren waarop studenten hun eigen taakgedrag bijhouden. Zelfmonitoring-interventies draaien vaak om het definiëren van een doelgedrag en vervolgens zelfbeoordeling en zelfregistratie hiervan. \autocite{Davis2016}
%Praktisch betekent dit bijvoorbeeld dat een leerling bij elke onderbroken werkactiviteit of afronding van een taak op een formulier markeert of hij doelgedrag heeft vertoond. In klassenonderzoek lieten deelnemers weten of ze ‘on task’ waren wanneer een signaalklokje ging, en noteerden dit op papier. \autocite{Davis2016}
%Uit meta-analyses blijkt dat deze ‘zelfopname’ essentieel is: effectieve zelfmonitoring-interventies bevatten altijd een component waarin leerlingen hun gedrag vastleggen. \autocite{Davis2016}
%Dergelijke papieren checklists of reflectiekaarten zijn eenvoudig in te zetten: de leerling streept bijvoorbeeld af wanneer hij een studietaak heeft voltooid of noteert korte reflecties als “waar liep ik vast?”. Dit verbetert het bewustzijn van studiegedrag en nodigt uit tot bijsturing.

%Visuele planning en timers: (Algemene tip) Klassieke niet-digitale hulpmiddelen zoals visuele weekplanners, taakstroken of fysieke timers kunnen structuur bieden. 
%Door studieactiviteiten in een overzichtelijk schema te plaatsen en tijdblokken af te bakenen, wordt het plannen en bijhouden eenvoudiger. 
%Hoewel minder onderzocht, adviseren vele leermethoden het gebruik van posters, schema’s of papieren tijdsplanners om studenten te helpen hun schema te volgen en eraan herinnerd te worden aan pauzes of deadlines.


%Digitale hulpmiddelen

%I-Connect: een app voor smartphone/tablet die deelnemers via visuele cues herinnert aan zelfmonitoring. De app toont op ingestelde tijden een herinnering (‘visual prompt’) waarmee de gebruiker aangeeft of hij/zij bezig is met de taak (bijv. “Ja” of “Nee”). Onderzoek laat zien dat I-Connect deelnemers helpt om meer ‘on-task’ gedrag te vertonen. Deelnemers klikken bij elk signaal op de yes/no-knop op het scherm om hun voortgang te registreren. \autocite{Li2023}
%I-Connect blijkt vooral effectief voor studenten met ASS die baat hebben bij concrete, herhaalde prikkels en eigen registratie van taakbetrokkenheid.

%Habitica: een gamified takenlijst-app die dagelijkse taken en studieactiviteiten in een rollenspel giet. De gebruiker maakt een karakter aan en verdient punten of beloningen als taken voltooid worden, en verliest punten als ze worden gemist. \autocite{Lieder2024} 
%Dit kan motivatie en zelfmonitoring bevorderen doordat studenten hun voortgang ‘spelenderwijs’ bijhouden. Onderzoek wijst wel uit dat Habitica’s puntensysteem niet altijd optimaal is: minder dan de helft van de gebruikers vindt de beloningen passend, en velen ervaren tegenstrijdige effecten door de willekeurige puntentoekenning. \autocite{Lieder2024} 
%Habitica blijft populair voor het motiveren van dagelijkse routines en kan leerlingen stimuleren om to-do’s consistent af te vinken, maar een zorgvuldige instellingen van doelen en beloningen is nodig voor maximale effectiviteit.

%Ace Your Self-Study: een speciaal ontwikkelde leerapp die cognitieve en metacognitieve strategieën promoot tijdens zelfstudie. \autocite{Baars2022}
%De app begeleidt studenten stap voor stap: voor een studeersessie maken zij een studieplan (taakkeuze, strategieën, doelen en tijd) en tijdens het studeren helpt de app om deze planning terug te halen. Na afloop volgt een reflectie. Door deze fasen (planning–uitvoering–reflectie) te koppelen, stimuleert de app studenten actief planning, monitoring en reflectie over hun leerproces. \autocite{Baars2022}
%Zo kan een student bijvoorbeeld vragen selecteren (summarizing, zelftesten, etc.) en een timer instellen; de app herinnert hen vervolgens aan de geplande pauzes of vraagt bij afronding om zelfevaluatie. Onderzoek beschrijft dat de app 20 evidence-based studievaardigheden aanbiedt met uitleg en video’s, en daarmee expliciet stuurt op zelfregulatie. \autocite{Baars2022}
%%Ace Your Self-Study ondersteunt dus het zelfmonitoringsproces door studenten aan te sporen hun leeractiviteiten bewust te plannen en achteraf te evalueren.\autocite{Baars2022}

%self-monitor-habit-changer: Een app waarmee gebruikers (bijv. studenten) zichzelf op vaste tijdstippen kunnen ‘zelfmonitoren’. 
%Je stelt in welke vraag je aan jezelf wilt stellen (zoals “Ben ik bezig met de taak?”), hoe vaak dit moet gebeuren en hoe lang de sessie duurt. 
%Bij elk signaal rapporteer je jouw gedrag, wat kan helpen om on‑task‑tijd te vergroten door bewuste check‑ins en reflectie. 
%De app ondersteunt geluid‑, trilling‑ of visuele herinneringen en kan meerdere taken per leerling instellen. Dit type zelfmonitoring is effectief gebleken in gedragsverandering, vooral voor taken waarbij frequente bewuste heroriëntatie belangrijk is. \autocite{SelfMonitorHabitChangerAppStore}

%Selbstlernen:De Selbstlernen.app wordt beschreven als een platform gericht op het bevorderen van zelfgereguleerd leren en het versterken van zelfregulatiecompetenties bij jonge mensen, met een focus op onderwijsinhoudelijke toepassing en inclusie. 
%Het project bevindt zich in de ontwikkelingsfase met een prototypische implementatie en beoogt interactieve hulpmiddelen en digitale diagnostiek te bieden in een iteratief, participatief ontwikkelingsproces. \autocite{SelbstlernenAppInklusionNetwork2025}

%Forest: Forest is een focus‑ en concentratie‑app die gamification gebruikt om afleidingen te beperken. Je “plant” een virtuele boom die groeit zolang je gefocust blijft op je taak — verlaat je de app, dan sterft de boom. 
%Hierdoor stimuleert Forest langere periodes van ononderbroken aandacht en bevordert het een gevoel van verantwoordelijkheid en productiviteit. 
%Naast de focus‑timer kun je via statistieken je focusgewoonten volgen, en er is een optie om met virtuele munten echte bomen te planten via een partnerschap met milieu‑organisaties. \autocite{ForestAppSiho2025}

%TickTick:TickTick is een alles‑in‑één productiviteits‑ en taakbeheertool die takenlijsten, kalender, gewoontetracker en een Pomodoro‑focus‑timer combineert. In TickTick kun je:

%%taken organiseren en prioriteren,

%gewoontes instellen en bijhouden,
%focussessies timen en visualiseren,

%je productiviteit statistisch analyseren.
%Het is cross‑platform beschikbaar en synchroniseert taken, herinneringen en gewoontes tussen apparaten. Hiermee ondersteunt TickTick planning, monitoring en voortgang in je dagelijkse routines en studieactiviteiten. \autocite{TickTickWebsite2026}

%Focusverse: Focusverse wordt beschreven als een gamified multiplayer Pomodoro timer waarmee gebruikers online samen focus‑sessies kunnen doen (“Focus with Friends”) en tijdens deze sessies beloningen zoals Focúmon verzamelen om motivatie en betrokkenheid te verhogen. \autocite{FocusverseProductHunt2025}

%Kortom: Er bestaan zowel digitale als analoge hulpmiddelen om zelfmonitoring bij zelfregulerend leren te ondersteunen. 
%Digitale apps als I-Connect, Habitica en Ace Your Self-Study stimuleren het bijhouden en plannen van taken door herinneringen en beloningen. 
%Niet-digitale middelen zoals planners, leerlogboeken en checklists geven studenten een concrete manier om doelen en voortgang op papier te registreren. 
%Door deze tools toe te passen, kan de zwakke planning en monitoring bij studenten met ASS versterkt worden, wat bijdraagt aan effectiever zelfgestuurd leren.

%Hier beschrijf je de \emph{state-of-the-art} rondom je gekozen onderzoeksdomein, d.w.z.\ een inleidende, doorlopende tekst over het onderzoeksdomein van je bachelorproef. Je steunt daarbij heel sterk op de professionele \emph{vakliteratuur}, en niet zozeer op populariserende teksten voor een breed publiek. Wat is de huidige stand van zaken in dit domein, en wat zijn nog eventuele open vragen (die misschien de aanleiding waren tot je onderzoeksvraag!)?

%Je mag de titel van deze sectie ook aanpassen (literatuurstudie, stand van zaken, enz.). Zijn er al gelijkaardige onderzoeken gevoerd? Wat concluderen ze? Wat is het verschil met jouw onderzoek?

%Verwijs bij elke introductie van een term of bewering over het domein naar de vakliteratuur, bijvoorbeeld~\autocite{WatsonBrossHuffman2021}! Denk zeker goed na welke werken je refereert en waarom.

%Draag zorg voor correcte literatuurverwijzingen! Een bronvermelding hoort thuis \emph{binnen} de zin waar je je op die bron baseert, dus niet er buiten! Maak meteen een verwijzing als je gebruik maakt van een bron. Doe dit dus \emph{niet} aan het einde van een lange paragraaf. Baseer nooit teveel aansluitende tekst op eenzelfde bron.

%Als je informatie over bronnen verzamelt in JabRef, zorg er dan voor dat alle nodige info aanwezig is om de bron terug te vinden (zoals uitvoerig besproken in de lessen Research Methods).

% Voor literatuurverwijzingen zijn er twee belangrijke commando's:
% \autocite{KEY} => (Auteur, jaartal) Gebruik dit als de naam van de auteur
%   geen onderdeel is van de zin.
% \textcite{KEY} => Auteur (jaartal)  Gebruik dit als de auteursnaam wel een
%   functie heeft in de zin (bv. ``Uit onderzoek door Doll & Hill (1954) bleek
%   ...'')

%Je mag deze sectie nog verder onderverdelen in subsecties als dit de structuur van de tekst kan verduidelijken.

%---------- Methodologie ------------------------------------------------------
\section{Methodologie}%
\label{sec:methodologie}


Dit onderzoek hanteert een ontwerpgerichte en vergelijkende onderzoeksaanpak, met als doel het identificeren en uitwerken van een digitale oplossing die zelfmonitoring binnen zelfregulerend leren bij studenten met ASS ondersteunt. 
De methodologie bestaat uit vijf opeenvolgende fasen, gaande van requirementsanalyse tot de ontwikkeling en evaluatie van een Proof of Concept (PoC).
Fasen van de vergelijkende studie

\subsection{Requirementsanalyse}

De eerste fase van het onderzoek bestaat uit het verzamelen en structureren van requirements waaraan een geschikte digitale oplossing moet voldoen. Deze requirements worden afgeleid uit:
\begin{itemize}

\item de probleemstelling en onderzoeksvraag

\item relevante wetenschappelijke literatuur

\item noden van de doelgroepen (bv. studenten, begeleiders, zorgprofessionals)
\end{itemize}

De geïdentificeerde requirements worden geclassificeerd aan de hand van de MoSCoW-methode:
\begin{itemize}

\item Must-have: essentiële vereisten waaraan een oplossing moet voldoen om bruikbaar te zij

\item Should-have: belangrijke vereisten die de werking aanzienlijk verbeteren

\item Could-have: optionele vereisten die een meerwaarde bieden

\item Won’t-have: vereisten die buiten de scope van dit onderzoek vallen
\end{itemize}

Het resultaat van deze fase is een gestructureerde lijst van functionele en niet-functionele requirements, geordend volgens prioriteit en belanghebbenden.

\subsection{Opstellen van een long list}

In de tweede fase wordt een long list opgesteld van bestaande alternatieven die potentieel relevant zijn voor het onderzoeksprobleem.
Deze lijst wordt samengesteld op basis van een systematisch literatuuronderzoek, aangevuld met:
\begin{itemize}

\item academische publicaties

\item bestaande commerciële en niet-commerciële tools

\item eerder ontwikkelde onderzoeksprototypes
\end{itemize}

In deze fase worden geen alternatieven uitgesloten. Het doel is een zo volledig mogelijk overzicht te krijgen van alle bestaande oplossingen die theoretisch of praktisch aansluiten bij de vooropgestelde requirements.

Het resultaat is een uitgebreide lijst van geïdentificeerde alternatieven, zonder beoordeling of filtering.

\subsection{Selectie van een short list}

In de derde fase wordt de long list gereduceerd tot een short list van de meest relevante alternatieven.
De selectie gebeurt door:
\begin{itemize}

\item de alternatieven te toetsen aan de opgestelde requirements

\item een vergelijkende matrix op te stellen waarin elk alternatief wordt geëvalueerd per criterium

\item per criterium een score toe te kennen, wat resulteert in een kwantitatieve vergelijking
\end{itemize}

Op basis van deze scores worden één of meerdere alternatieven geselecteerd die:
\begin{itemize}

\item het best aansluiten bij de must- en should-requirements

\item voldoende potentieel hebben om het probleem uit de onderzoeksvraag aan te pakken
\end{itemize}

Deze short list vormt de basis voor verdere analyse en ontwikkeling.

\subsection{Ontwikkeling van een Proof of Concept}

Op basis van de resultaten van de vergelijkende studie wordt één alternatief gekozen voor verdere uitwerking in de vorm van een Proof of Concept (PoC).
Het doel van deze PoC is te valideren of het geselecteerde alternatief effectief een oplossing kan bieden voor het gestelde probleem.

De ontwikkeling van de PoC volgt opnieuw de MoSCoW-methode:
\begin{itemize}

\item Must: kernfunctionaliteiten die noodzakelijk zijn om de werking van de oplossing te demonstreren

\item Should en Could: bijkomende functionaliteiten die, indien haalbaar, worden geïmplementeerd

\item Won’t: functionaliteiten die expliciet niet worden opgenomen in de PoC
\end{itemize}

De PoC bestaat uit een werkend prototype van een ICT-oplossing, voldoende uitgebreid om:
\begin{itemize}

\item de belangrijkste functionaliteiten te testen

\item de toepasbaarheid van de geselecteerde requirements te evalueren
\end{itemize}

\subsection{Conclusies en aanbevelingen}

In de laatste fase worden de resultaten van zowel de vergelijkende studie als de Proof of Concept samengebracht.
Op basis hiervan wordt:
\begin{itemize}

\item een onderbouwde aanbeveling geformuleerd over het meest geschikte alternatief

\item geëvalueerd in welke mate de PoC voldoet aan de vooropgestelde requirements

\item hiaten en beperkingen geïdentificeerd in de huidige oplossing
\end{itemize}

Tot slot worden aanbevelingen geformuleerd voor:
\begin{itemize}

\item verdere optimalisatie of uitbreiding van de oplossing

\item toekomstig onderzoek binnen dit domein
\end{itemize}


%Hier beschrijf je hoe je van plan bent het onderzoek te voeren. Welke onderzoekstechniek ga je toepassen om elk van je onderzoeksvragen te beantwoorden? Gebruik je hiervoor literatuurstudie, interviews met belanghebbenden (bv.~voor requirements-analyse), experimenten, simulaties, vergelijkende studie, risico-analyse, PoC, \ldots?

%Valt je onderwerp onder één van de typische soorten bachelorproeven die besproken zijn in de lessen Research Methods (bv.\ vergelijkende studie of risico-analyse)? Zorg er dan ook voor dat we duidelijk de verschillende stappen terug vinden die we verwachten in dit soort onderzoek!

%Vermijd onderzoekstechnieken die geen objectieve, meetbare resultaten kunnen opleveren. Enquêtes, bijvoorbeeld, zijn voor een bachelorproef informatica meestal \textbf{niet geschikt}. De antwoorden zijn eerder meningen dan feiten en in de praktijk blijkt het ook bijzonder moeilijk om voldoende respondenten te vinden. Studenten die een enquête willen voeren, hebben meestal ook geen goede definitie van de populatie, waardoor ook niet kan aangetoond worden dat eventuele resultaten representatief zijn.

%Uit dit onderdeel moet duidelijk naar voor komen dat je bachelorproef ook technisch voldoen\-de diepgang zal bevatten. Het zou niet kloppen als een bachelorproef informatica ook door bv.\ een student marketing zou kunnen uitgevoerd worden.

%Je beschrijft ook al welke tools (hardware, software, diensten, \ldots) je denkt hiervoor te gebruiken of te ontwikkelen.

%Probeer ook een tijdschatting te maken. Hoe lang zal je met elke fase van je onderzoek bezig zijn en wat zijn de concrete \emph{deliverables} in elke fase?

%---------- Verwachte resultaten ----------------------------------------------
\section{Verwacht resultaat, conclusie}%
\label{sec:verwachte_resultaten}

\subsection{Conclusie}

Dit onderzoek richt zich op het verkennen van ondersteunende hulpmiddelen voor zelfmonitoring binnen zelfregulerend leren bij studenten met ASS. Uit de literatuur blijkt dat beperkingen in executieve functies, zoals planning, organisatie, werkgeheugen en flexibiliteit, het zelfstandig leren en zelfmonitoring aanzienlijk bemoeilijken. Analoge en digitale hulpmiddelen bieden mogelijkheden om deze uitdagingen te adresseren, waarbij digitale tools extra functionaliteiten bieden zoals real-time prompts, automatische registratie en visuele feedback die aansluiten bij de behoeften van de doelgroep.

Op basis van een systematische requirementsanalyse en een vergelijkende beoordeling van bestaande oplossingen is een selectie gemaakt van veelbelovende alternatieven. Deze selectie vormt de basis voor de ontwikkeling van een Proof of Concept, dat dient als exploratief prototype om de toepasbaarheid van de geïdentificeerde functionaliteiten te onderzoeken. De PoC biedt een concreet middel om inzicht te krijgen in hoe digitale ondersteuning zelfmonitoring bij ASS-studenten kan faciliteren.

\subsection{Verwacht resultaat}

Het verwachte resultaat van dit onderzoek is een werkend prototype dat de kernfunctionaliteiten van het gekozen alternatief illustreert en experimenteel toont hoe digitale ondersteuning kan worden ingezet voor zelfmonitoring binnen zelfregulerend leren. Concreet levert dit:
\begin{itemize}

\item Een overzichtelijk en gebruiksvriendelijk prototype dat de belangrijkste requirements ondersteunt

\item Een inzicht in de praktische toepasbaarheid van digitale hulpmiddelen voor studenten met ASS

\item Handvatten voor toekomstige iteraties of verder onderzoek, bijvoorbeeld om functies te optimaliseren of toe te voegen

\item Een gestructureerde evaluatie van hoe de geselecteerde functionaliteiten aansluiten bij de behoeften en uitdagingen van de doelgroep

\end{itemize}

Het resultaat biedt daarmee een solide basis voor vervolgonderzoek en ontwikkeling, zonder dat het direct effect op studiegedrag of academische prestaties claimt, we verwachten dat het proof-of-concept van de geselecteerde oplossing een merkbare positieve impact zal hebben op zelfmonitoring en zelfregulatie bij studenten met ASS. Indien de bevindingen anders uitvallen dan verwacht, biedt dat waardevolle inzichten in aanvullende uitdagingen of onvoorziene factoren. Uiteindelijk levert dit onderzoek concrete handvatten op voor het ondersteunen van zelfgestuurd leren van een doelgroep die hier door karakteristieke zwaktes extra moeilijk toe in staat is.
%Hier beschrijf je welke resultaten je verwacht. Als je metingen en simulaties uitvoert, kan je hier al mock-ups maken van de grafieken samen met de verwachte conclusies. Benoem zeker al je assen en de onderdelen van de grafiek die je gaat gebruiken. Dit zorgt ervoor dat je concreet weet welk soort data je moet verzamelen en hoe je die moet meten.

%Wat heeft de doelgroep van je onderzoek aan het resultaat? Op welke manier zorgt jouw bachelorproef voor een meerwaarde?

%Hier beschrijf je wat je verwacht uit je onderzoek, met de motivatie waarom. Het is \textbf{niet} erg indien uit je onderzoek andere resultaten en conclusies vloeien dan dat je hier beschrijft: het is dan juist interessant om te onderzoeken waarom jouw hypothesen niet overeenkomen met de resultaten.

\printbibliography[heading=bibintoc]

\end{document}