%---------- Inleiding ---------------------------------------------------------

% TODO: Is dit voorstel gebaseerd op een paper van Research Methods die je
% vorig jaar hebt ingediend? Heb je daarbij eventueel samengewerkt met een
% andere student?
% Zo ja, haal dan de tekst hieronder uit commentaar en pas aan.

%\paragraph{Opmerking}

% Dit voorstel is gebaseerd op het onderzoeksvoorstel dat werd geschreven in het
% kader van het vak Research Methods dat ik (vorig/dit) academiejaar heb
% uitgewerkt (met medesturent VOORNAAM NAAM als mede-auteur).
% 

\section{Inleiding}%
\label{sec:inleiding}

Studenten met de diagnose autisme (Autisme Spectrum Stoornis ASS) ervaren in het hoger onderwijs vaak moeilijkheden op het gebied van executieve functies. Deze mentale functies zijn essentieel voor het reguleren en controleren van andere cognitieve processen, met name tijdens complexe handelingen. Voorbeelden van belangrijke executieve functies zijn selectieve aandacht aanwenden om processen te selecteren om een doel te bereiken, het plannen van taken, schakelen tussen taken, inhibitie en zelfmonitoring. 
Door hun verminderde cognitieve flexibiliteit vertrouwen studenten met ASS doorgaans op vertrouwde, vaste routines. Wanneer zich onverwachte situaties voordoen, hebben zij moeite om over te schakelen naar een alternatieve strategie, waardoor handelingen voor hen meer tijd vragen in vergelijking met neurotypische studenten. In combinatie met overprikkeling, onvoorspelbaarheid en sociale uitdagingen leidt dit vaak tot extreme vermoeidheid, spanning en stress, wat de studieresultaten negatief beïnvloedt. Daarnaast rapporteren studenten met ASS vaker concentratieproblemen, uitstelgedrag en cognitieve overbelasting -  factoren die  een aanzienlijke impact hebben op hun studievoortgang en algemeen welbevinden.
Samenvattend blijkt uit onderzoek dat studenten met ASS extra ondersteuning nodig hebben, vooral bij het flexibel bijsturen van hun studietaken.
Zelfmonitoring, of het controleren of de uitvoering van de taak correct verloopt, is een kernonderdeel van zelfregulerend studeren. Concreet betekent dit dat de student bewust aandacht besteedt aan specifieke aspecten van zijn eigen leerproces, zoals controleren of hij nog volgens de vooropgestelde planning werkt en het identificeren van obstakels. Op deze manier vergroten studenten met ASS hun inzicht in effectieve leerstrategieën. Dit kan worden ondersteund met technieken en tools zoals checklists, logboeken of korte reflectievragen die helpen bepalen of de focus behouden blijft. Dergelijke strategieën versterken de zelfbewustheid en helpen studenten efficiënter  en doelgerichter studeren.
Een mogelijke oplossing voor de uitdagingen waarmee studenten met ASS kampen op het gebied van executieve functies - specifiek zelf-monitoring - is het inzetten van digitale tools. Dit leidt tot de centrale onderzoeksvraag "Hoe kan een digitale tool studenten met ASS ondersteunen bij het verbeteren van zelfmonitoring binnen zelfregulerend leren in het hoger onderwijs?". Om deze vraag te beantwoorden, zal een vergelijkende studie uitgevoerd worden naar I-Connect \& Habitica. Daarnaast wordt er een Proof-Of-Concept ontwikkeld dat studenten op een motiverende manier ondersteunt bij het toepassen van zelfmonitoring binnen hun eigen leermethode. 
%Waarover zal je bachelorproef gaan? Introduceer het thema en zorg dat volgende zaken zeker duidelijk aanwezig zijn:

%\begin{itemize}
%  \item kaderen thema
%  \item de doelgroep
%  \item de probleemstelling en (centrale) onderzoeksvraag
%  \item de onderzoeksdoelstelling
%\end{itemize}

%Denk er aan: een typische bachelorproef is \textit{toegepast onderzoek}, wat betekent dat je start vanuit een concrete probleemsituatie in bedrijfscontext, een \textbf{casus}. Het is belangrijk om je onderwerp goed af te bakenen: je gaat voor die \textit{ene specifieke probleemsituatie} op zoek naar een goede oplossing, op basis van de huidige kennis in het vakgebied.

%De doelgroep moet ook concreet en duidelijk zijn, dus geen algemene of vaag gedefinieerde groepen zoals \emph{bedrijven}, \emph{developers}, \emph{Vlamingen}, enz. Je richt je in elk geval op it-professionals, een bachelorproef is geen populariserende tekst. Eén specifiek bedrijf (die te maken hebben met een concrete probleemsituatie) is dus beter dan \emph{bedrijven} in het algemeen.

%Formuleer duidelijk de onderzoeksvraag! De begeleiders lezen nog steeds te veel voorstellen waarin we geen onderzoeksvraag terugvinden.

%Schrijf ook iets over de doelstelling. Wat zie je als het concrete eindresultaat van je onderzoek, naast de uitgeschreven scriptie? Is het een proof-of-concept, een rapport met aanbevelingen, \ldots Met welk eindresultaat kan je je bachelorproef als een succes beschouwen?

%---------- Stand van zaken ---------------------------------------------------

\section{Literatuurstudie}%
\label{sec:literatuurstudie}
Uit onderzoek blijkt dat jongvolwassen met hoog functionerend autisme moeilijkheden ervaren met zelfregulerende vaardigheden\autocite{dijkhuis2017self}. Dit vormt een uitdaging aangezien zelfmonitoring een cruciaal onderdeel is van zelfregulerend leren. Het onderzoek van Isaacson en Fujita (2006) toont aan dat studenten die hun eigen begrip, voortgang en prestaties accuraat kunnen monitoren, significant beter in staat zijn hun leerstrategieën bij te sturen en daardoor hogere academische resultaten behalen. Een gebrek aan effectieve zelfmonitoring beperkt dus rechtstreeks de mogelijkheid om het leerproces te optimaliseren en succesvoller te presteren in een onderwijscontext. \autocite{IsaacsonFujita2006}

De huidige ondersteuning voor zelfmonitoring bij studenten met ASS bestaat voornamelijk uit expliciete, visuele en sterk gestructureerde interventies, die zelfmonitorende stappen concreet en voorspelbaar maken. Studies tonen dat visuele schema’s, checklists en zelfmonitoringskaarten de taakgerichtheid en zelfregulatie verbeteren \autocite{Ganz2008}. Daarnaast blijken modelling en video- modelling studenten te helpen om monitoringsgedrag te internaliseren. Deze methoden hebben een bewezen effect bij het aanleren van complexe zelfregulerende stappen \autocite{BelliniAkullian2007}.

Ook technologische middelen kunnen zelfmonitoring ondersteunen. Een voorbeeld hiervan is I-Connect, een digitaal zelfmonitoringsysteem dat studenten op vaste momenten laat controleren of zij een bepaald doelgedag hebben bereikt. Het gebruik van de app, vaak in combinatie met een beloningssysteem, leidt tot een aanzienlijke toename in taakgericht gedrag. \autocite{DelVecchioCroslandFuller2024}  Een meta-analyse bevestigt dat I-Connect in verschillende studiepopulaties sterke functionele effecten laat zien: studenten met een ondersteuningsbehoefte vertonen consequent meer on-task gedrag wanneer ze I-Connect gebruiken. \autocite{ScheibelZimmermanWills2023}
Een tweede voorbeeld is Habitica, een gamified taken- en gewoontebeheerder app, waarin dagelijkse taken worden voorgesteld als een opdracht. Gebruikers verdienen punten of virtuele beloningen wanneer zij taken afronden. Het idee is gebaseerd op gamificatie: het toepassen van spelelementen (punten, levels, badges en beloningen)op niet-spelcontexten om motivatie en regelmaat in gedrag te verhogen. \autocite{DiefenbachMuessig2019}

%Hier beschrijf je de \emph{state-of-the-art} rondom je gekozen onderzoeksdomein, d.w.z.\ een inleidende, doorlopende tekst over het onderzoeksdomein van je bachelorproef. Je steunt daarbij heel sterk op de professionele \emph{vakliteratuur}, en niet zozeer op populariserende teksten voor een breed publiek. Wat is de huidige stand van zaken in dit domein, en wat zijn nog eventuele open vragen (die misschien de aanleiding waren tot je onderzoeksvraag!)?

%Je mag de titel van deze sectie ook aanpassen (literatuurstudie, stand van zaken, enz.). Zijn er al gelijkaardige onderzoeken gevoerd? Wat concluderen ze? Wat is het verschil met jouw onderzoek?

%Verwijs bij elke introductie van een term of bewering over het domein naar de vakliteratuur, bijvoorbeeld~\autocite{WatsonBrossHuffman2021}! Denk zeker goed na welke werken je refereert en waarom.

%Draag zorg voor correcte literatuurverwijzingen! Een bronvermelding hoort thuis \emph{binnen} de zin waar je je op die bron baseert, dus niet er buiten! Maak meteen een verwijzing als je gebruik maakt van een bron. Doe dit dus \emph{niet} aan het einde van een lange paragraaf. Baseer nooit teveel aansluitende tekst op eenzelfde bron.

%Als je informatie over bronnen verzamelt in JabRef, zorg er dan voor dat alle nodige info aanwezig is om de bron terug te vinden (zoals uitvoerig besproken in de lessen Research Methods).

% Voor literatuurverwijzingen zijn er twee belangrijke commando's:
% \autocite{KEY} => (Auteur, jaartal) Gebruik dit als de naam van de auteur
%   geen onderdeel is van de zin.
% \textcite{KEY} => Auteur (jaartal)  Gebruik dit als de auteursnaam wel een
%   functie heeft in de zin (bv. ``Uit onderzoek door Doll & Hill (1954) bleek
%   ...'')

%Je mag deze sectie nog verder onderverdelen in subsecties als dit de structuur van de tekst kan verduidelijken.

%---------- Methodologie ------------------------------------------------------
\section{Methodologie}%
\label{sec:methodologie}
Dit onderzoek start met een vergelijkende studie van bestaande tools. De beoordeling van elke tool gebeurt aan de hand van een set criteria, die is afgeleid uit een vooraf uitgevoerde requirementsanalyse. Elke tool krijgt een score op basis van een aantal vereisten die ze vervullen, die een kwantitatieve vergelijking van hun geschiktheid mogelijk maakt.

Op basis van deze requirements wordt een Proof of Concept (PoC) ontwikkeld. Het ontwikkelproces volgt de MoSCoW-methode. Deze methode implementeert functionaliteiten volgens hun prioriteit (Must, Should, Could, Won’t). Deze PoC vormt een praktische demonstratie hoe de digitale tool de geselecteerde criteria toepast.
%Hier beschrijf je hoe je van plan bent het onderzoek te voeren. Welke onderzoekstechniek ga je toepassen om elk van je onderzoeksvragen te beantwoorden? Gebruik je hiervoor literatuurstudie, interviews met belanghebbenden (bv.~voor requirements-analyse), experimenten, simulaties, vergelijkende studie, risico-analyse, PoC, \ldots?

%Valt je onderwerp onder één van de typische soorten bachelorproeven die besproken zijn in de lessen Research Methods (bv.\ vergelijkende studie of risico-analyse)? Zorg er dan ook voor dat we duidelijk de verschillende stappen terug vinden die we verwachten in dit soort onderzoek!

%Vermijd onderzoekstechnieken die geen objectieve, meetbare resultaten kunnen opleveren. Enquêtes, bijvoorbeeld, zijn voor een bachelorproef informatica meestal \textbf{niet geschikt}. De antwoorden zijn eerder meningen dan feiten en in de praktijk blijkt het ook bijzonder moeilijk om voldoende respondenten te vinden. Studenten die een enquête willen voeren, hebben meestal ook geen goede definitie van de populatie, waardoor ook niet kan aangetoond worden dat eventuele resultaten representatief zijn.

%Uit dit onderdeel moet duidelijk naar voor komen dat je bachelorproef ook technisch voldoen\-de diepgang zal bevatten. Het zou niet kloppen als een bachelorproef informatica ook door bv.\ een student marketing zou kunnen uitgevoerd worden.

%Je beschrijft ook al welke tools (hardware, software, diensten, \ldots) je denkt hiervoor te gebruiken of te ontwikkelen.

%Probeer ook een tijdschatting te maken. Hoe lang zal je met elke fase van je onderzoek bezig zijn en wat zijn de concrete \emph{deliverables} in elke fase?

%---------- Verwachte resultaten ----------------------------------------------
\section{Verwacht resultaat, conclusie}%
\label{sec:verwachte_resultaten}

De hypothese is dat de vergelijking van I-Connect en Habitica zal aantonen dat digitale zelfmonitoringtools studenten met ASS op verschillende manieren kunnen ondersteunen: via gestructureerde voortgangsregistratie, visuele feedback en motivatie via gamification. Voor studenten met ASS in het hoger onderwijs, die vaak moeite hebben met het inschatten van hun eigen kennis, voortgang, en taakgerichtheid zijn vooral gestructureerde en visueel ondersteunde functies waardevol. 
%Hier beschrijf je welke resultaten je verwacht. Als je metingen en simulaties uitvoert, kan je hier al mock-ups maken van de grafieken samen met de verwachte conclusies. Benoem zeker al je assen en de onderdelen van de grafiek die je gaat gebruiken. Dit zorgt ervoor dat je concreet weet welk soort data je moet verzamelen en hoe je die moet meten.

%Wat heeft de doelgroep van je onderzoek aan het resultaat? Op welke manier zorgt jouw bachelorproef voor een meerwaarde?

%Hier beschrijf je wat je verwacht uit je onderzoek, met de motivatie waarom. Het is \textbf{niet} erg indien uit je onderzoek andere resultaten en conclusies vloeien dan dat je hier beschrijft: het is dan juist interessant om te onderzoeken waarom jouw hypothesen niet overeenkomen met de resultaten.


