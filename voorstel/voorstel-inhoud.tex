%---------- Inleiding ---------------------------------------------------------

% TODO: Is dit voorstel gebaseerd op een paper van Research Methods die je
% vorig jaar hebt ingediend? Heb je daarbij eventueel samengewerkt met een
% andere student?
% Zo ja, haal dan de tekst hieronder uit commentaar en pas aan.

%\paragraph{Opmerking}

% Dit voorstel is gebaseerd op het onderzoeksvoorstel dat werd geschreven in het
% kader van het vak Research Methods dat ik (vorig/dit) academiejaar heb
% uitgewerkt (met medesturent VOORNAAM NAAM als mede-auteur).
% 

\section{Inleiding}%
\label{sec:inleiding}
- Welke problemen ervaren jongvolassenen met ASS op vlak van executieve functies?

- Op welke manieren beperken uitdagingen in executieve functies bij studenten met ASS hun vermogen om hun volledige academische potentieel te benutten in het hoger onderwijs?

- welke uitdagingen hebben jongvolwassenen met ASS om zelfregulerend leren toe te passen.

- Welke specifieke kenmerken van ASS bemoeilijken het toepassen van zelfmonitoringstrategieën bij het studeren?

- Welke digitale hulpmiddelen bestaan er ter ondersteuning van zelfmonitoring bij zelfregulatie?

- Welke niet-digitale hulpmiddelen kunnen zelfmonitoring en zelfregulatie bevorderen?

- Hoe ondersteunen concrete tools als I-Connect, Habitica en Ace Your Self-Study het zelfmonitoringsproces?

Studenten met de diagnose autisme (Autisme Spectrum Stoornis ASS) ervaren in het hoger onderwijs vaak moeilijkheden op het gebied van executieve functies. 
Deze mentale functies zijn essentieel voor het reguleren en controleren van andere cognitieve processen, met name tijdens complexe handelingen. 
Voorbeelden van belangrijke executieve functies zijn selectieve aandacht aanwenden om processen te selecteren om een doel te bereiken, het plannen van taken, schakelen tussen taken, inhibitie en zelfmonitoring.%opsomming andere belangrijkere skills
Door hun verminderde cognitieve flexibiliteit vertrouwen studenten met ASS doorgaans op vertrouwde, vaste routines. 
Wanneer zich onverwachte situaties voordoen, hebben zij moeite om over te schakelen naar een alternatieve strategie, waardoor handelingen voor hen meer tijd vragen in vergelijking met neurotypische studenten. 
In combinatie met overprikkeling, onvoorspelbaarheid en sociale uitdagingen leidt dit vaak tot extreme vermoeidheid, spanning en stress, wat de studieresultaten negatief beïnvloedt. %noodzakelijk ?????
Daarnaast rapporteren studenten met ASS vaker concentratieproblemen, uitstelgedrag en cognitieve overbelasting -  factoren die een aanzienlijke impact hebben op hun studievoortgang en algemeen welbevinden.
Samenvattend blijkt uit onderzoek dat studenten met ASS extra ondersteuning nodig hebben, vooral bij het flexibel bijsturen van hun studietaken.
Zelfmonitoring, of het controleren of de uitvoering van de taak correct verloopt, is een kernonderdeel van zelfregulerend studeren. 
Concreet betekent dit dat de student bewust aandacht besteedt aan specifieke aspecten van zijn eigen leerproces, zoals controleren of hij nog volgens de vooropgestelde planning werkt en het identificeren van obstakels. 
Op deze manier vergroten studenten met ASS hun inzicht in effectieve leerstrategieën. 
Dit kan worden ondersteund met technieken en tools zoals checklists, logboeken of korte reflectievragen die helpen bepalen of de focus behouden blijft. %betere opsomming
Dergelijke strategieën versterken de zelfbewustheid en helpen studenten efficiënter en doelgerichter studeren.
Een mogelijke oplossing voor de uitdagingen waarmee studenten met ASS kampen op het gebied van executieve functies - specifiek zelf-monitoring - is het inzetten van digitale tools. 
Dit leidt tot de centrale onderzoeksvraag "Hoe kan een digitale tool studenten met ASS ondersteunen bij het verbeteren van zelfmonitoring binnen zelfregulerend leren in het hoger onderwijs?". Om deze vraag te beantwoorden, zal een vergelijkende studie uitgevoerd worden naar I-Connect \& Habitica. Daarnaast wordt er een Proof-Of-Concept ontwikkeld dat studenten op een motiverende manier ondersteunt bij het toepassen van zelfmonitoring binnen hun eigen leermethode. 
%Waarover zal je bachelorproef gaan? Introduceer het thema en zorg dat volgende zaken zeker duidelijk aanwezig zijn:

%\begin{itemize}
%  \item kaderen thema
%  \item de doelgroep
%  \item de probleemstelling en (centrale) onderzoeksvraag
%  \item de onderzoeksdoelstelling
%\end{itemize}

%Denk er aan: een typische bachelorproef is \textit{toegepast onderzoek}, wat betekent dat je start vanuit een concrete probleemsituatie in bedrijfscontext, een \textbf{casus}. Het is belangrijk om je onderwerp goed af te bakenen: je gaat voor die \textit{ene specifieke probleemsituatie} op zoek naar een goede oplossing, op basis van de huidige kennis in het vakgebied.

%De doelgroep moet ook concreet en duidelijk zijn, dus geen algemene of vaag gedefinieerde groepen zoals \emph{bedrijven}, \emph{developers}, \emph{Vlamingen}, enz. Je richt je in elk geval op it-professionals, een bachelorproef is geen populariserende tekst. Eén specifiek bedrijf (die te maken hebben met een concrete probleemsituatie) is dus beter dan \emph{bedrijven} in het algemeen.

%Formuleer duidelijk de onderzoeksvraag! De begeleiders lezen nog steeds te veel voorstellen waarin we geen onderzoeksvraag terugvinden.

%Schrijf ook iets over de doelstelling. Wat zie je als het concrete eindresultaat van je onderzoek, naast de uitgeschreven scriptie? Is het een proof-of-concept, een rapport met aanbevelingen, \ldots Met welk eindresultaat kan je je bachelorproef als een succes beschouwen?

%---------- Stand van zaken ---------------------------------------------------

\section{Literatuurstudie}%
\label{sec:literatuurstudie}
Er zal eerst gekeken worden naar de huidige stand van zaken voor studenten met autisme.

Jongvolwassenen met autisme hebben vaak moeite met verschillende executieve functies. Onderzoek toont aan dat sterke ASS‐kenmerken correleren met zwakker functioneren op vrijwel alle EF‐domeinen. \autocite{VanEsch2022} 
Vooral op planning en organisatie loopt het vaak slecht: ze vergeten snel tijd en kunnen niet goed inschatten wat er moet gebeuren. Ook flexibiliteit is beperkt: ASS’ers volgen meestal één vaste strategie en hebben moeite om van aanpak te wisselen. \autocite{meerman2020inclusief}
Verder zien we tekorten in werkgeheugen en taakinitiatie (het op gang komen met een taak) en in inhibitie (het onderdrukken van impulsen). \autocite{duncan2022school}
Uit zelfrapportages blijkt dat jonge volwassenen met hoogfunctionerend autisme in het dagelijks leven aanzienlijk meer problemen ervaren met executieve functies dan hun niet-autistische leeftijdsgenoten. \autocite{dijkhuis2017selfregulation}
Hieruit kan geconcludeerd worden dat deficits in plannen, organiseren, timemanagement, flexibiliteit en zelfmonitoring veelvoorkomend zijn bij ASS.

In het hoger onderwijs zijn juist goede executieve vaardigheden cruciaal voor succes. Studenten moeten zelfstandig studeren, inplannen, deadlines halen, materiaal organiseren en bij groepsopdrachten initiatief nemen. \autocite{dijkhuis2017selfregulation}
Als deze EF’s beperkt zijn, leidt dat tot concrete studieproblemen. Zo kan gebrek aan planningsvaardigheid ertoe leiden dat colleges of opdrachten gemist worden, en problemen met timemanagement zorgen dat tentamens en projecten niet op tijd afkomen. \autocite{meerman2020inclusief}
Uit een review blijkt dat leerlingen met ASS vaak last hebben met ‘organizing materials’, ‘prioritizing assignments’, ‘studying effectively’ en het opbreken van grote taken, precies essentiële leergedragingen in het voortgezet en hoger onderwijs. \autocite{duncan2022school}
Als gevolg hiervan blijven veel studenten met ASS 2–3 jaar achter bij hun neurotypische klasgenoten in studieprestaties. 
Kortom: problemen met executief functioneren vertalen zich in haperende studiegewoonten en slechtere studieresultaten. Sterk ontwikkelde executieve functies zijn juist wezenlijk voor studenten met ASS in het hoger onderwijs. \autocite{duncan2022school}

Door de tekort komingen in executieve functies hebben ze problemen met vaardigheden zoals plannen organiseren, timemanagement, flexibiliteit en zelfmonitoring wat een negatieve impact heeft op hun studieresultaten. 
Een techniek om deze vaardigheden te verbeteren is Zelf regulerend leren.

Zelfregulerend leren vergt dat studenten hun eigen leerproces kunnen plannen, monitoren en bijsturen. Jongeren met ASS ervaren hierbij vaak knelpunten. Ze hebben beperkte metacognitieve vaardigheden, ze vinden het moeilijk hun eigen begrip te beoordelen of systematisch hun leervorderingen te volgen. \autocite{VanEsch2022}
In combinatie met verminderde planningscapaciteit en rigiditeit leidt dit ertoe dat ze weinig effectief zelfgestuurd leren. Ze stellen bijvoorbeeld vaker geen concrete doelen, laten studiestappen liggen en merken pas laat dat ze niet op schema liggen. Dökkerbächer en Bregulla (2024) benadrukken dat executieve functies (metacognitie, planning, inhibitie) de drijvende kracht zijn achter zelfregulatie. \autocite{duncan2022school}
Omdat deze bij ASS vaak minder sterk ontwikkeld zijn, wordt het inzetten van strategieën als tijdsplanning, vooruitgangscontrole en reflectie bemoeilijkt. Met andere woorden: de combinatie van zwakke planning, beperkte flexibiliteit en gebrekkig zelfbewustzijn maakt het voor jongeren met ASS extra moeilijk om gestructureerd en doelgericht te leren. \autocite{DörrenbächerUlrichBregulla2024}

Specifieke autistische kenmerken verergeren het probleem van zelfmonitoring bij het studeren. Door de beperkte cognitieve flexibiliteit en het sterrogende denkpatroon volgen ASS’ers starre routines: zij houden vast aan één strategie en passen deze niet snel aan als er iets misgaat. Zo vergeten ze bijvoorbeeld gemakkelijk om pauzes te nemen of om hun studie-uren bij te houden, waardoor planning faalt.  \autocite{meerman2020inclusief}
Daarnaast zorgt een zwakke centrale coherentie (focus op detail in plaats van het geheel) en verminderd inlevingsvermogen ervoor dat ze slecht inschatten hoe hun studiegedrag verloopt. Metacognitieve beperkingen betekenen dat ze vaak niet inzien dat iets niet begrepen is of dat ze achterlopen op schema. \autocite{kool2025metacognitie} 
Ook zelfevaluatie is problematisch: onderzoek bij HFASD toont dat problemen met “behavioral monitoring” en het bewust zijn van de gevolgen van eigen gedrag leiden tot achterblijvende resultaten. \autocite{dijkhuis2017selfregulation}
Kortom, autisme-kenmerken zoals starheid, beperkte zelfreflectie en gerichte detailverwerking bemoeilijken het toepassen van zelfmonitoringstrategieën bij het studeren (bv. tijd bijhouden, werk controleren of leerdoelen herzien) waardoor studenten met ASS vaak niet volledig profiteren van hun leeractiviteiten.

Wanneer er gekeken word naar zelfmonitoring skills dan is dit cruciaal onderdeel is van zelfregulerend leren. 
Het onderzoek van Isaacson en Fujita (2006) toont aan dat studenten die hun eigen begrip, voortgang en prestaties accuraat kunnen monitoren, significant beter in staat zijn hun leerstrategieën bij te sturen en daardoor hogere academische resultaten behalen. 
Een gebrek aan effectieve zelfmonitoring beperkt dus rechtstreeks de mogelijkheid om het leerproces te optimaliseren en succesvoller te presteren in een onderwijscontext. \autocite{IsaacsonFujita2006}

Studenten met ASS hebben vaak zwakkere executieve functies zoals plannen, organiseren, timemanagement, flexibiliteit en zelfmonitoring. 
Deze tekorten zorgen in het hoger onderwijs voor studieproblemen en mindere studieresultaten. 
Ook zelfregulerend leren verloopt hierdoor moeilijk, vooral omdat zelfmonitoring en metacognitie beperkt zijn. 
Daarom wordt in dit onderzoek gekeken naar oplossingen om zelfmonitoring binnen zelfregulerend leren te versterken bij studenten met ASS.




Niet-digitale hulpmiddelen

Papieren planners en leerdagboeken: Traditionele hulpmiddelen zoals papieren agenda’s, studieplanners of leerdagboeken helpen studenten om doelen en studie-taken uit te schrijven. In een leerdagboek (self-learning diary) noteert een student bijvoorbeeld wekelijks wat hij heeft gestudeerd, welke strategieën zijn gebruikt en in hoeverre doelen zijn behaald. Zulke zelfreflectie-instrumenten bevorderen zelfmonitoring doordat ze studenten bewust maken van hun planning en voortgang. \autocite{Cheng2025}
Onderzoek toont aan dat (digitale) leerdagboeken het stellen van leerdoelen en het bijhouden van de voortgang faciliteren – beide cruciale componenten van zelfregulatie. \autocite{Cheng2025}
Ook zonder app kunnen leerlingen bijgehouden schemaboekjes of checklist-formulieren gebruiken om doelen te stellen en aantekeningen te maken over hoever ze zijn gekomen.

Zelf-monitoringschecklists en reflectiekaarten: Een veelgebruikte analoge methode is het invullen van checklists of formulieren waarop studenten hun eigen taakgedrag bijhouden. Zelfmonitoring-interventies draaien vaak om het definiëren van een doelgedrag en vervolgens zelfbeoordeling en zelfregistratie hiervan. \autocite{Davis2016}
Praktisch betekent dit bijvoorbeeld dat een leerling bij elke onderbroken werkactiviteit of afronding van een taak op een formulier markeert of hij doelgedrag heeft vertoond. In klassenonderzoek lieten deelnemers weten of ze ‘on task’ waren wanneer een signaalklokje ging, en noteerden dit op papier. \autocite{Davis2016}
Uit meta-analyses blijkt dat deze ‘zelfopname’ essentieel is: effectieve zelfmonitoring-interventies bevatten altijd een component waarin leerlingen hun gedrag vastleggen. \autocite{Davis2016}
Dergelijke papieren checklists of reflectiekaarten zijn eenvoudig in te zetten: de leerling streept bijvoorbeeld af wanneer hij een studietaak heeft voltooid of noteert korte reflecties als “waar liep ik vast?”. Dit verbetert het bewustzijn van studiegedrag en nodigt uit tot bijsturing.

Visuele planning en timers: (Algemene tip) Klassieke niet-digitale hulpmiddelen zoals visuele weekplanners, taakstroken of fysieke timers kunnen structuur bieden. 
Door studieactiviteiten in een overzichtelijk schema te plaatsen en tijdblokken af te bakenen, wordt het plannen en bijhouden eenvoudiger. 
Hoewel minder onderzocht, adviseren vele leermethoden het gebruik van posters, schema’s of papieren tijdsplanners om studenten te helpen hun schema te volgen en eraan herinnerd te worden aan pauzes of deadlines.


Digitale hulpmiddelen

I-Connect: een app voor smartphone/tablet die deelnemers via visuele cues herinnert aan zelfmonitoring. De app toont op ingestelde tijden een herinnering (‘visual prompt’) waarmee de gebruiker aangeeft of hij/zij bezig is met de taak (bijv. “Ja” of “Nee”). Onderzoek laat zien dat I-Connect deelnemers helpt om meer ‘on-task’ gedrag te vertonen. Deelnemers klikken bij elk signaal op de yes/no-knop op het scherm om hun voortgang te registreren. \autocite{Li2023}
I-Connect blijkt vooral effectief voor studenten met ASS die baat hebben bij concrete, herhaalde prikkels en eigen registratie van taakbetrokkenheid.

Habitica: een gamified takenlijst-app die dagelijkse taken en studieactiviteiten in een rollenspel giet. De gebruiker maakt een karakter aan en verdient punten of beloningen als taken voltooid worden, en verliest punten als ze worden gemist. \autocite{Lieder2024} 
Dit kan motivatie en zelfmonitoring bevorderen doordat studenten hun voortgang ‘spelenderwijs’ bijhouden. Onderzoek wijst wel uit dat Habitica’s puntensysteem niet altijd optimaal is: minder dan de helft van de gebruikers vindt de beloningen passend, en velen ervaren tegenstrijdige effecten door de willekeurige puntentoekenning. \autocite{Lieder2024} 
Habitica blijft populair voor het motiveren van dagelijkse routines en kan leerlingen stimuleren om to-do’s consistent af te vinken, maar een zorgvuldige instellingen van doelen en beloningen is nodig voor maximale effectiviteit.

Ace Your Self-Study: een speciaal ontwikkelde leerapp die cognitieve en metacognitieve strategieën promoot tijdens zelfstudie. \autocite{Baars2022}
De app begeleidt studenten stap voor stap: voor een studeersessie maken zij een studieplan (taakkeuze, strategieën, doelen en tijd) en tijdens het studeren helpt de app om deze planning terug te halen. Na afloop volgt een reflectie. Door deze fasen (planning–uitvoering–reflectie) te koppelen, stimuleert de app studenten actief planning, monitoring en reflectie over hun leerproces. \autocite{Baars2022}
Zo kan een student bijvoorbeeld vragen selecteren (summarizing, zelftesten, etc.) en een timer instellen; de app herinnert hen vervolgens aan de geplande pauzes of vraagt bij afronding om zelfevaluatie. Onderzoek beschrijft dat de app 20 evidence-based studievaardigheden aanbiedt met uitleg en video’s, en daarmee expliciet stuurt op zelfregulatie. \autocite{Baars2022}
Ace Your Self-Study ondersteunt dus het zelfmonitoringsproces door studenten aan te sporen hun leeractiviteiten bewust te plannen en achteraf te evalueren.\autocite{Baars2022}



Kortom: Er bestaan zowel digitale als analoge hulpmiddelen om zelfmonitoring bij zelfregulerend leren te ondersteunen. 
Digitale apps als I-Connect, Habitica en Ace Your Self-Study stimuleren het bijhouden en plannen van taken door herinneringen en beloningen. 
Niet-digitale middelen zoals planners, leerlogboeken en checklists geven studenten een concrete manier om doelen en voortgang op papier te registreren. 
Door deze tools toe te passen, kan de zwakke planning en monitoring bij studenten met ASS versterkt worden, wat bijdraagt aan effectiever zelfgestuurd leren.



De huidige ondersteuning van zelfmonitoring binnen zelfgereguleerd leren (ZRL) bij studenten met autismespectrumstoornissen (ASS) bestaat uit digitale tools en niet-digitale tools. 
Deze aanpakken zijn gericht op het externaliseren van zelfmonitorende processen om deze voorspelbaar en hanteerbaar te maken. 
Veelgebruikte tools zijn visuele schema’s, papieren checklists en zelfmonitoringskaarten, die studenten ondersteunen bij het plannen, volgen en evalueren van hun leeractiviteiten. 
Onderzoek toont aan dat dergelijke visuele en tastbare hulpmiddelen bijdragen aan verhoogde taakgerichtheid en verbeterde zelfregulatie \autocite{Ganz2008, ReidTrout2005}.

Daarnaast wordt modelling, waaronder (video-)modelling, ingezet om zelfmonitoringsgedrag expliciet voor te doen en te internaliseren. 
Deze methoden blijken effectief bij het aanleren van complexe, sequentiële zelfregulerende vaardigheden bij leerlingen en studenten met ASS \autocite{BelliniAkullian2007, Hume2014}. 
Over het geheel genomen benadrukt de literatuur dat niet-digitale ondersteuning van zelfmonitoring binnen ZRL vooral steunt op externe structuur en begeleide instructie, wat de zelfstandige toepassing van zelfmonitoringsvaardigheden deels afhankelijk maakt van voortdurende ondersteuning.





De huidige ondersteuning voor zelfmonitoring bij studenten met ASS bestaat voornamelijk uit expliciete, visuele en sterk gestructureerde interventies, die zelfmonitorende stappen concreet en voorspelbaar maken. 
Studies tonen dat visuele schema’s, checklists en zelfmonitoringskaarten de taakgerichtheid en zelfregulatie verbeteren \autocite{Ganz2008}. 
Daarnaast blijken modelling en video- modelling studenten te helpen om monitoringsgedrag te internaliseren. 
Deze methoden hebben een bewezen effect bij het aanleren van complexe zelfregulerende stappen \autocite{BelliniAkullian2007}.

Ook technologische middelen kunnen zelfmonitoring ondersteunen. 
Een voorbeeld hiervan is I-Connect, een digitaal zelfmonitoringsysteem dat studenten op vaste momenten laat controleren of zij een bepaald doelgedag hebben bereikt. 
Het gebruik van de app, vaak in combinatie met een beloningssysteem, leidt tot een aanzienlijke toename in taakgericht gedrag. \autocite{DelVecchioCroslandFuller2024}  
Een meta-analyse bevestigt dat I-Connect in verschillende studiepopulaties sterke functionele effecten laat zien: studenten met een ondersteuningsbehoefte vertonen consequent meer on-task gedrag wanneer ze I-Connect gebruiken. \autocite{ScheibelZimmermanWills2023}
Een tweede voorbeeld is Habitica, een gamified taken- en gewoontebeheerder app, waarin dagelijkse taken worden voorgesteld als een opdracht. 
Gebruikers verdienen punten of virtuele beloningen wanneer zij taken afronden. 
Het idee is gebaseerd op gamificatie: het toepassen van spelelementen (punten, levels, badges en beloningen)op niet-spelcontexten om motivatie en regelmaat in gedrag te verhogen. \autocite{DiefenbachMuessig2019}

%Hier beschrijf je de \emph{state-of-the-art} rondom je gekozen onderzoeksdomein, d.w.z.\ een inleidende, doorlopende tekst over het onderzoeksdomein van je bachelorproef. Je steunt daarbij heel sterk op de professionele \emph{vakliteratuur}, en niet zozeer op populariserende teksten voor een breed publiek. Wat is de huidige stand van zaken in dit domein, en wat zijn nog eventuele open vragen (die misschien de aanleiding waren tot je onderzoeksvraag!)?

%Je mag de titel van deze sectie ook aanpassen (literatuurstudie, stand van zaken, enz.). Zijn er al gelijkaardige onderzoeken gevoerd? Wat concluderen ze? Wat is het verschil met jouw onderzoek?

%Verwijs bij elke introductie van een term of bewering over het domein naar de vakliteratuur, bijvoorbeeld~\autocite{WatsonBrossHuffman2021}! Denk zeker goed na welke werken je refereert en waarom.

%Draag zorg voor correcte literatuurverwijzingen! Een bronvermelding hoort thuis \emph{binnen} de zin waar je je op die bron baseert, dus niet er buiten! Maak meteen een verwijzing als je gebruik maakt van een bron. Doe dit dus \emph{niet} aan het einde van een lange paragraaf. Baseer nooit teveel aansluitende tekst op eenzelfde bron.

%Als je informatie over bronnen verzamelt in JabRef, zorg er dan voor dat alle nodige info aanwezig is om de bron terug te vinden (zoals uitvoerig besproken in de lessen Research Methods).

% Voor literatuurverwijzingen zijn er twee belangrijke commando's:
% \autocite{KEY} => (Auteur, jaartal) Gebruik dit als de naam van de auteur
%   geen onderdeel is van de zin.
% \textcite{KEY} => Auteur (jaartal)  Gebruik dit als de auteursnaam wel een
%   functie heeft in de zin (bv. ``Uit onderzoek door Doll & Hill (1954) bleek
%   ...'')

%Je mag deze sectie nog verder onderverdelen in subsecties als dit de structuur van de tekst kan verduidelijken.

%---------- Methodologie ------------------------------------------------------
\section{Methodologie}%
\label{sec:methodologie}
Het onderzoek bestaat uit twee hoofdonderdelen: (1) een vergelijkende analyse van bestaande zelfmonitoring-tools, en (2) de ontwikkeling van een webgebaseerde Proof of Concept (PoC) met prioritering van functionaliteiten.

Eerst wordt een set beoordelingscriteria opgesteld, gebaseerd op literatuur over leertechnologie voor ASS.

Mogelijkheid tot registratie van taakuitvoering (voltooid / niet voltooid)

Registratie van studietijd per taak

Registratie van on-task gedrag / focus

Historiek en overzicht van studieactiviteiten

Mogelijkheid tot zelfevaluatie na een taak

Ondersteuning van reflectie op studiegedrag

Ondersteuning bij doelen formuleren

Ondersteuning bij planning van studietaken

Monitoring tijdens uitvoering (reminders, prompts)

Reflectie na afloop van een studiesessie

Feedback die aanzet tot bijsturing van strategieën

Intuïtieve en eenvoudige interface

Beperkt aantal handelingen per taak

Lage cognitieve belasting

Consistente structuur en navigatie

Afwezigheid van overprikkelende elementen

Visuele weergave van taken en planning

Grafische voortgangsindicatoren

Gebruik van kleur, iconen of schema’s

Visuele tijdsondersteuning (timers, tijdsblokken)

Directe feedback op gedrag of taakstatus

Samenvattende feedback over voortgang

Feedback die begrijpelijk en concreet is

Positieve feedback bij gewenst gedrag

Aanpasbare doelen en taken

Aanpasbare reminders of prompts

Flexibiliteit om eigen studiemethode te volgen

Ondersteuning van vaste routines

Motivatie via zichtbare voortgang

Eventuele beloningen of gamificatie

Beperkt risico op demotivatie bij falen

Onderbouwd door wetenschappelijke studies

Aangetoond effect op zelfmonitoring of on-task gedrag

Relevantie voor studenten met ASS
%Hier beschrijf je hoe je van plan bent het onderzoek te voeren. Welke onderzoekstechniek ga je toepassen om elk van je onderzoeksvragen te beantwoorden? Gebruik je hiervoor literatuurstudie, interviews met belanghebbenden (bv.~voor requirements-analyse), experimenten, simulaties, vergelijkende studie, risico-analyse, PoC, \ldots?

%Valt je onderwerp onder één van de typische soorten bachelorproeven die besproken zijn in de lessen Research Methods (bv.\ vergelijkende studie of risico-analyse)? Zorg er dan ook voor dat we duidelijk de verschillende stappen terug vinden die we verwachten in dit soort onderzoek!

%Vermijd onderzoekstechnieken die geen objectieve, meetbare resultaten kunnen opleveren. Enquêtes, bijvoorbeeld, zijn voor een bachelorproef informatica meestal \textbf{niet geschikt}. De antwoorden zijn eerder meningen dan feiten en in de praktijk blijkt het ook bijzonder moeilijk om voldoende respondenten te vinden. Studenten die een enquête willen voeren, hebben meestal ook geen goede definitie van de populatie, waardoor ook niet kan aangetoond worden dat eventuele resultaten representatief zijn.

%Uit dit onderdeel moet duidelijk naar voor komen dat je bachelorproef ook technisch voldoen\-de diepgang zal bevatten. Het zou niet kloppen als een bachelorproef informatica ook door bv.\ een student marketing zou kunnen uitgevoerd worden.

%Je beschrijft ook al welke tools (hardware, software, diensten, \ldots) je denkt hiervoor te gebruiken of te ontwikkelen.

%Probeer ook een tijdschatting te maken. Hoe lang zal je met elke fase van je onderzoek bezig zijn en wat zijn de concrete \emph{deliverables} in elke fase?

%---------- Verwachte resultaten ----------------------------------------------
\section{Verwacht resultaat, conclusie}%
\label{sec:verwachte_resultaten}

De hypothese is dat de vergelijking van I-Connect en Habitica zal aantonen dat digitale zelfmonitoringtools studenten met ASS op verschillende manieren kunnen ondersteunen: via gestructureerde voortgangsregistratie, visuele feedback en motivatie via gamification. Voor studenten met ASS in het hoger onderwijs, die vaak moeite hebben met het inschatten van hun eigen kennis, voortgang, en taakgerichtheid zijn vooral gestructureerde en visueel ondersteunde functies waardevol. 
%Hier beschrijf je welke resultaten je verwacht. Als je metingen en simulaties uitvoert, kan je hier al mock-ups maken van de grafieken samen met de verwachte conclusies. Benoem zeker al je assen en de onderdelen van de grafiek die je gaat gebruiken. Dit zorgt ervoor dat je concreet weet welk soort data je moet verzamelen en hoe je die moet meten.

%Wat heeft de doelgroep van je onderzoek aan het resultaat? Op welke manier zorgt jouw bachelorproef voor een meerwaarde?

%Hier beschrijf je wat je verwacht uit je onderzoek, met de motivatie waarom. Het is \textbf{niet} erg indien uit je onderzoek andere resultaten en conclusies vloeien dan dat je hier beschrijft: het is dan juist interessant om te onderzoeken waarom jouw hypothesen niet overeenkomen met de resultaten.